% !TeX root = ../Tex/main.tex

\section*{Conclusion}
The puzzle motivating this paper was deliberately simple: two elephant-human encounters occur in broadly similar environments, yet they generate sharply different behavioural trajectories. In one occasion, the elephant's threat is forceful but calibrated; in the other, violence is followed by a striking form of care.

\vspace*{1em}
Standard signalling logic is largely successful in explaining why an elephant might issue a credible threat and why a human might respond cautiously. What it does not readily explain is the combination of escalation and subsequent protection--an outcome that looks puzzling from the standpoint of canonical rational-agent models. The central claim advanced here is that the apparent inconsistency largely reflects omitted elements of the model: the relevant "shock" can be internal, and the relevant constraint can be affective.

\vspace*{1em}
The causal-inference lens is useful. Treating the two episodes as potential outcomes of a shared underlying interaction makes the research design explicit: the cases serve as mutual counterfactuals, and the goal is to isolate which latent factor, omitted in standard game-theoretic frameworks, generates the divergence. Within this framing, surprise functions as an informational discontinuity that plausibly shifts the elephant's perception of threat and thus the equilibrium-relevant mapping from signals to actions. Crucially, however, surprise alone cannot account for the post-attack behaviour. That second step motivates the introduction of empathy as a regulator: a force that can attenuate aggression, reshape the "intensity" of signals, and generate assistance once the situation is reinterpreted.

\vspace*{1em}
This conceptual move has a representational payoff. Instead of treating emotions as noise that perturbs otherwise coherent strategic choice, the paper sketches a way to represent emotions as structured, competing influences over cognition. The adaptation of the vote-buying framework provides a particularly transparent metaphor: thoughts play the role of passive "voters", while emotions act as active players allocating scarce intensity to secure agenda-setting power--here, the determination of behaviour. 

\vspace*{1em}
In equilibrium, the elephant's observable trajectory corresponds to a stable emotional configuration: neither emotion can improve its outcome given the other's allocation. Read in this way, "attack, then care" is not a contradiction so much as the observable consequence of a reallocation of cognitive control after a misinterpretation is recognised.

The broader implication is that many phenomena labeled as "shocks" in political science and international relations may be mischaracterised when treated as purely exogenous disturbances.
A framework that makes these mechanisms explicit can therefore complement familiar strategic models, not by abandoning rationality, but by widening the set of state variables that rational choice is allowed to condition on.

\vspace*{1em}
At the same time, the limits of the present contribution should be stated plainly. A fuller treatment would require specifying payoffs, information structures, and equilibrium selection more rigorously. Future work could therefore proceed by tightening the formal model, including the relationship between surprise and empathy as distinct but interacting state variables.

\vspace*{1em}
The main message is modest: when game theory explains puzzling conduct by invoking "surprise", some times it may be pointing towards a missing internal mechanism. Making that mechanism explicit--rather than treating it as residual noise--can turn an ad hoc rationalisation into a tractable extension of formal modelling, one that is better aligned with the texture of real strategic behaviour.

%------- Acknowledgements -------
\subsubsection*{Acknowledgements}
Artificial intelligence-based tools were employed solely to improve linguistic clarity and grammar. No AI system contributed to the development of the research questions, theoretical framework or conclusions presented in this paper.