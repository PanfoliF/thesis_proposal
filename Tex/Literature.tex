% !TeX root = ../Tex/main.tex

\section{Literature Review and Limitations}

\subsection{A Review of Models of Conflicts}
Further exploration of the conflict-games literature may either support or qualify the analysis presented so far. In any case, already-established models provide a useful point of comparison. Some of the papers that stand out include \textcite{schellingStrategyConflict1960, Schelling1966}. On the other hand, it may be of interest to focus on \textcite{mckelveyQuantalResponseEquilibria1995} to further develop the design.\

\subsubsection{A Review of Models of Conflicts with a Focus on Bargaining}
One prominent strand of the literature examines on bargaining models of conflict. Key contributions to this field number \textcite{Fearon1996, Fearon_1995, Fearon_1998} and \textcite{schellingStrategyConflict1960, Schelling1966}.
\textcite{reiterExploringBargainingModel2003} offers an in-depth review of the field. He provides an historical perspective and discusses more recent works. Additionally, he summarises the most important ideas.

\vspace*{1em}
\textcite{reiterExploringBargainingModel2003} says: "James Fearon prominently contributed to this scholarship with his 1995 paper "Rationalist Explanations for War." He usefully highlighted the point that if states agreed on the outcome of a possible war, they could probably avoid war. Central to Fearon's paper is the importance of focusing on states' disagreement over their capabilities and/or resolve. [\dots] Fearon uses a bargaining model to argue that if two states in dispute know the outcome of a possible war, they should in general prefer to reach a deal that would reflect the hypothetical postwar political settlement, rather than fight, reach that same settlement, and also suffer the costs of war. This view assumes that fighting itself is costly--that the belligerent always suffers some negative utility, no matter how the issue at stake is settled. In fact, this assumption becomes necessary when one asserts that two states would always prefer to reach a bargain without fighting rather than fight and then reach the same bargain. Within his bargaining model, Fearon develops three conditions under which war is possible."

\vspace*{1em}
To broaden the overview, some other salient articles comprise
\textcite{Kirshner01092000, wagnerBargainingWar2000, wittmanHowWarEnds1979, smithBargainingNatureWar2004, filsonBargainingFightingImpact2004}.

\subsection{Limits of Theoretical Models}
This discussion is necessarily selective and cannot provide a comprehensive assessment of the formal-modelling literature. In the previous pages I have presented a personal re-elaboration along with a brief summary of the relevant literature in this area. This is certainly not enough to assess the state of the models of this essay. However, the exposition I provide should not be taken as comprehensive, since there is no such thing as \textit{a single correct or definitive modelling framework.}

In this respect, it is possible to further expand the discussion on limitations. The literature offers some very valuable contributions, among others, I discuss:
\textcite{ashworthTheoryCredibilityIntegrating2021, brauningerTheoryBuildingCausal2020, alma991001906059707991}. These essays presents different ways to combine empirical and theoretical researc together with some insightful interpretations of the meaning and scope of models.

\subsection{Connection between Game Theory and Biology}
There are many important applications of game theory in biology. Particularly influential examples include: \textcite{smithLogicAnimalConflict1973, axelrodEvolutionCooperation1981}.
Future work may shed light on more recent and equally notable applications.
A vast literature covers the topic of evolutionary games as well: \textcite{kandoriLearningMutationLong1993, taylorEvolutionaryStableStrategies1978, hofbauerEvolutionaryGamesPopulation2003, nowakFiveRulesEvolution2006}. In particular, \textcite[chap 12]{hofbauerEvolutionaryGamesPopulation2003} offers a clear treatise of powerful analytical tools together with applications of these game-theoretic tools in biology.

Unfortunately, a more detailed discussion of these works must be postponed.